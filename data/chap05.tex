% !TeX root = ../main.tex

\chapter{总结与展望}

\section{总结}

本文基于NeRF技术开发了一套光照分解管线,旨在解决数字资产与渲染技术之间的耦合问题。
在第3章中,本文结合研究背景和现有方法分析了解耦的目标并选定使用的技术,
设计并实现了数字资产解耦管线。在第4章中,本文根据第3章实验结果中的错误,开展了光照表示的研究。

在数字资产解耦管线的实现方面,本文基于NeRF光照分解相关研究展开工作。
针对现有研究无法将管线输出直接用于传统渲染器以及下游编辑任务的问题,本文采用DMTet结合法线纹理的几何表示方式,
配合正则项的设计,实现了能够直接输出显式网格体的管线。该管线能够无缝对接传统渲染器,扩展了NeRF的下游应用场景。
对于现有研究只能使用单一工作流进行分解的限制,本文设计了MLP纹理,这种设计能够灵活地从多个通道中划分出需要的反射属性,
而且允许管线能够输出纹理贴图。最终,定性定量的实验证明了本文完成了既定目标,成功使用NeRF解除数字资产与渲染技术的耦合。

在光照表示的研究方面,本文从IBL技术对阴影表达能力的欠缺入手,引入了即时渲染中常见的光照组合方式。基于条件输入以及
自动编码器,本文设计了新颖的神经光照表示SaNL。SaNL能够通过阴影强度结合直接光照与间接光照,
在定量定性实验中展现出优异的性能,使用SaNL的光照分解管线在多个数据集上优于其它研究。
本文还通过消融实验进一步揭示了SaNL的工作原理,证明了SaNL的有效性。

通过以上工作,本文不仅实现了解耦数字资产与渲染技术的目标,在实际应用上展现出价值,并且为光照表示的研究提出了新的可能,
具备理论创新性。

\section{展望}

尽管取得上述成果,但仍有诸多方面值得进一步探索。在未来可能的研究方向中,工业应用和学术扩展是本文工作的重点。

虽然现在的解耦管线已经能够实现直接将输出用于传统渲染器,但是,具体应用上仍有诸多不便。例如,
管线的输入数据还需要估计相机位姿,虽然众多已有方法能够满足这一需求,但是这仍然需要额外的步骤。
未来,数字资产的解耦工作可以进一步地扩展为端到端的管线,直接嵌入传统资产编辑软件,进一步提升实用性。

本文认为,光照分解管线是一个极具潜力的研究方向。虽然本文专注于对数字资产进行分解,但是还有更多方法的输出可以作为本文管线的输入。
比如能够用文本生成图像的人工智能模型,该类模型能够直接输出高质量二维图片,在经过光照分解管线后,可以将产物直接用于渲染引擎,
能够大幅提升数字资产的制作效率。

除了本文现有的应用场景以外,本文管线也可以进一步扩展为按照更复杂的着色模型和工作流进行分解,
比如散射、折射等着色模型。另外,现有即时渲染领域经常使用各类定制的纹理贴图或着色技术以满足特殊效果,
如果本文能够对这些特殊效果进行分解,则不仅能提高特殊效果的开发效率,还能作为新效果的探索和快速验证。

通过以上这些可能的研究方向可以得出,本文的研究仍有极大的潜力。
本文期待能够在渲染和数字资产制作领域发挥更大的作用,助力行业的发展与探索。
