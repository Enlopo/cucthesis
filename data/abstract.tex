% !TeX root = ../main.tex

% 中英文摘要和关键字

\begin{abstract}
  
  % 研究生论文由于篇幅过长,通常可以划分几点
  % 主要工作内容如下:1. 2. 3.
  % \begin{enumerate}
  %   \item 
  %   \item 
  %   \item 
  % \end{enumerate}

  % 关键词用“英文逗号”分隔,输出时会自动处理为正确的分隔符,通常使用分号
  \thusetup{
    keywords = {NeRF, 三维重建, 可微渲染, NeRF分解, 渲染管线},
  }
\end{abstract}

% 使用 DeepL 翻译:https://www.deepl.com/
% 使用 Grammarly 检查语法:https://www.grammarly.com/
\begin{abstract*}
  % 留空行缩进

  % Use comma as separator when inputting
  \thusetup{
    keywords* = {NeRF, 3D Reconstruction, Differentiable Rendering, NeRF Decomposition, Render Pipeline},
  }
\end{abstract*}
