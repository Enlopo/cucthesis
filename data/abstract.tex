% !TeX root = ../main.tex

% 中英文摘要和关键字

\begin{abstract}

  渲染(Rendering)是根据场景信息和着色模型生成二维图像的过程,是计算机图形
学的核心问题和最终目标。
场景通常以数字模型资产(Digital Model Asset)的形式存在,这些资产包含网
格模型以及纹理贴图,用以表示几何形状、反射属性等多种信息。不同的渲染技术则对
资产的内容和格式提出了不同要求,二者紧密耦合。这种耦合使得资产一旦制作完成,
资产便与渲染技术形成了绑定,使得资产在渲染技术迭代或变化时难以兼容,增加了项
目开发的成本和风险。

为了解决这一问题,本文提出了一种新颖的解决方案——基于NeRF的数字资产解耦管线。
该管线基于神经辐射场(Neural Radiance Fields,NeRF)的光照分解研究,从多张二维图像中建立场景的几何形状及反射属性,
以网格体和纹理贴图的形式导出,实现了将任意数字资产转换为指定渲染技术兼容的格式,完成解耦的目标。

NeRF光照分解研究中,主要分为几何表示方法、光照表示方法以及反射属性三个部分。基于以上三个部分,本文主要工作内容如下:
\begin{enumerate}
  \item 对于隐式几何表示无法直接导出网格体的不足,本文采用了深度行进四面体(Deep Marching Tetrahedra,DMTet)作为
  混合几何表示方法,结合了SDF和网格体的优点,使得管线能够直接输出显式网格体,无缝对接传统渲染器与下游编辑任务。
  \item 现有的可微渲染研究中,光照表示无法有效处理阴影。本文通过结合条件输入,设计了一宗新颖的阴影感知神经光照表示
  (Shadow-aware Neural Light,SaNL),并收集了对应的数据进行训练。实验证明,本文的光照表示方法能够正确还原阴影处的数据,
  在几何重建及光照分解效果上优于同类其它工作。
  \item 针对本文管线能够以任意渲染技术进行分解的需求,本文使用多层感知机(Multilayer Perceptron, MLP)纹理来表示表面反射属性。
  这种实现方式为管线提供了足够的灵活性,满足了本文使用任意渲染技术的目标。
\end{enumerate}
  \thusetup{
    keywords = {NeRF, 三维重建, 可微渲染, NeRF分解, 渲染管线},
  }
\end{abstract}

% 使用 DeepL 翻译:https://www.deepl.com/
% 使用 Grammarly 检查语法:https://www.grammarly.com/
\begin{abstract*}

  Rendering generates two-dimensional images from scene information and shading models, 
  representing the core task of computer graphics. Typically, 
  scenes are described as Digital Model Assets consisting of mesh models and textures, 
  encoding geometry and reflection properties. Different rendering methods have specific asset requirements, 
  leading to tight coupling, compatibility issues, and increased development costs when rendering techniques evolve.

  To address this, this paper introduces a NeRF-based digital asset decoupling pipeline. 
  Leveraging Neural Radiance Fields (NeRF) for lighting decomposition, 
  the pipeline reconstructs geometry and reflection properties from multiple images, 
  exporting assets as meshes and textures. 
  This allows arbitrary digital assets to be converted into formats compatible with various rendering techniques, 
  achieving decoupling.
  
  This paper's key contributions based on NeRF lighting decomposition include:
  
  \begin{enumerate}
  \item Adopting Deep Marching Tetrahedra (DMTet) to overcome limitations of implicit geometry by directly exporting explicit meshes compatible with traditional renderers.
  \item Proposing a Shadow-aware Neural Light (SaNL) representation using conditional inputs, effectively capturing shadows and surpassing existing methods in reconstruction and lighting decomposition.
  \item Employing Multilayer Perceptron (MLP) textures to flexibly represent surface reflection properties, ensuring compatibility with various rendering techniques.
  \end{enumerate}
  
  % Use comma as separator when inputting
  \thusetup{
    keywords* = {NeRF, 3D Reconstruction, Differentiable Rendering, NeRF Decomposition, Render Pipeline},
  }
\end{abstract*}
