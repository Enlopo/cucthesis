% !TeX root = ../main.tex

% 中英文摘要和关键字

\begin{abstract}

渲染(Rendering)是根据场景信息和着色模型生成二维图像的过程,是计算机图形学的核心问题和最终目标。
场景通常以数字模型资产(Digital Model Asset)的形式存在,这些资产包含网格模型以及纹理贴图,
用以表示几何形状、反射属性等多种信息。不同的渲染技术则对资产的内容和格式提出了不同要求,
二者紧密耦合。这种耦合使得数字资产与特定渲染技术深度绑定,当需要升级渲染管线时,现有资产往往需要重新制作,
增加了项目开发的成本和风险。

为了解决这一问题,本文提出了一种新颖的解决方案。
本文基于神经辐射场(Neural Radiance Fields,NeRF)的光照分解研究,从多张二维图像中建立场景的几何形状及反射属性,
以网格体和纹理贴图的形式导出,构建了基于NeRF的数字资产解耦管线,实现了将任意数字资产转换为指定渲染技术兼容的格式。

NeRF光照分解研究中,主要分为几何表示方法、光照表示方法以及反射属性三个部分。基于以上三个部分,本文主要工作内容如下:

对于隐式几何表示无法直接导出网格体的不足,本文采用了深度行进四面体(Deep Marching Tetrahedra,DMTet)作为
混合几何表示方法。DMTet兼具神经隐式表示和显式表示的优点,配合本文针对网格体重建质量而设计的正则项和训练策略,
使得管线能够直接输出显式网格体,无缝对接传统渲染器与下游编辑任务。

其次,现有研究中的光照表示无法有效处理阴影。本文通过结合条件输入,设计了一种新颖的阴影感知神经光照表示
(Shadow-Aware Neural Light,SANL),并构建了包含真实的阴影和几何形状的专用数据集。随后,本文通过完善的定量定性实验
以及完整管线实验证明了SANL的有效性和优越性。

在反射属性部分,针对本文管线能够以任意渲染技术进行分解的需求,本文使用多层感知机(Multi-Layer Perceptron,MLP)纹理来表示表面反射属性,
在估计。这种实现方式为管线提供了足够的灵活性,不仅满足了本文使用任意渲染技术的目标,
同时为本文通过法线细化几何形状、建模阴影的扩展提供了技术基础。

基于以上工作及实验,本文不仅达成了对数字资产进行解耦的目标,并且光照分解效果优异。
相较于同类工作,本文管线的峰值信噪比及结构相似性指标提升了\%,学习感知图像相似度降低了【】\%,并在形状重建指标倒角距离提升了\%。

\thusetup{
  keywords = {NeRF, 三维重建, 可微渲染, NeRF分解, 渲染管线},
}
\end{abstract}

% 使用 DeepL 翻译:https://www.deepl.com/
% 使用 Grammarly 检查语法:https://www.grammarly.com/
\begin{abstract*}

  Rendering generates two-dimensional images from scene information and shading models, 
  representing the core task of computer graphics. Typically, 
  scenes are described as Digital Model Assets consisting of mesh models and textures, 
  encoding geometry and reflection properties. Different rendering methods have specific asset requirements, 
  leading to tight coupling, compatibility issues, and increased development costs when rendering techniques evolve.

  To address this, this paper introduces a NeRF-based digital asset decoupling pipeline. 
  Leveraging Neural Radiance Fields (NeRF) for lighting decomposition, 
  the pipeline reconstructs geometry and reflection properties from multiple images, 
  exporting assets as meshes and textures. 
  This allows arbitrary digital assets to be converted into formats compatible with various rendering techniques, 
  achieving decoupling.
  
  This paper's key contributions based on NeRF lighting decomposition include:
  
  Adopting Deep Marching Tetrahedra (DMTet) to overcome limitations of implicit geometry by directly exporting explicit meshes compatible with traditional renderers.
  
  Proposing a Shadow-aware Neural Light (SANL) representation using conditional inputs, effectively capturing shadows and surpassing existing methods in reconstruction and lighting decomposition.
  
  Employing Multi-Layer Perceptron (MLP) textures to flexibly represent surface reflection properties, ensuring compatibility with various rendering techniques.
  
  % Use comma as separator when inputting
  \thusetup{
    keywords* = {NeRF, 3D Reconstruction, Differentiable Rendering, NeRF Decomposition, Render Pipeline},
  }
\end{abstract*}
